% Options for packages loaded elsewhere
\PassOptionsToPackage{unicode}{hyperref}
\PassOptionsToPackage{hyphens}{url}
%
\documentclass[
]{article}
\usepackage{lmodern}
\usepackage{amssymb,amsmath}
\usepackage{ifxetex,ifluatex}
\ifnum 0\ifxetex 1\fi\ifluatex 1\fi=0 % if pdftex
  \usepackage[T1]{fontenc}
  \usepackage[utf8]{inputenc}
  \usepackage{textcomp} % provide euro and other symbols
\else % if luatex or xetex
  \usepackage{unicode-math}
  \defaultfontfeatures{Scale=MatchLowercase}
  \defaultfontfeatures[\rmfamily]{Ligatures=TeX,Scale=1}
\fi
% Use upquote if available, for straight quotes in verbatim environments
\IfFileExists{upquote.sty}{\usepackage{upquote}}{}
\IfFileExists{microtype.sty}{% use microtype if available
  \usepackage[]{microtype}
  \UseMicrotypeSet[protrusion]{basicmath} % disable protrusion for tt fonts
}{}
\makeatletter
\@ifundefined{KOMAClassName}{% if non-KOMA class
  \IfFileExists{parskip.sty}{%
    \usepackage{parskip}
  }{% else
    \setlength{\parindent}{0pt}
    \setlength{\parskip}{6pt plus 2pt minus 1pt}}
}{% if KOMA class
  \KOMAoptions{parskip=half}}
\makeatother
\usepackage{xcolor}
\IfFileExists{xurl.sty}{\usepackage{xurl}}{} % add URL line breaks if available
\IfFileExists{bookmark.sty}{\usepackage{bookmark}}{\usepackage{hyperref}}
\hypersetup{
  hidelinks,
  pdfcreator={LaTeX via pandoc}}
\urlstyle{same} % disable monospaced font for URLs
\setlength{\emergencystretch}{3em} % prevent overfull lines
\providecommand{\tightlist}{%
  \setlength{\itemsep}{0pt}\setlength{\parskip}{0pt}}
\setcounter{secnumdepth}{-\maxdimen} % remove section numbering

\author{}
\date{}

\begin{document}

\textbf{References}

\begin{quote}
Al Bkhetan, Z., Zobel, J., Kowalczyk, A., Verspoor, K., \& Goudey, B.
(2019). Exploring effective approaches for haplotype block phasing.
\emph{BMC Bioinformatics}, \emph{20}(1), 540.

Berg, J. J., Harpak, A., Sinnott-Armstrong, N., Joergensen, A. M.,
Mostafavi, H., Field, Y., Boyle, E. A., Zhang, X., Racimo, F.,
Pritchard, J. K., \& Coop, G. (2019). Reduced signal for polygenic
adaptation of height in UK Biobank. \emph{ELife}, \emph{8}.
https://doi.org/10.7554/eLife.39725

Bhat, J. A., Yu, D., Bohra, A., Ganie, S. A., \& Varshney, R. K. (2021).
Features and applications of haplotypes in crop breeding.
\emph{Communications Biology}, \emph{4}(1), 1--12.

Brandt, D. Y. C., Wei, X., Deng, Y., Vaughn, A. H., \& Nielsen, R.
(2021). Evaluation of methods for the inference of ancestral
recombination graphs. In \emph{bioRxiv} (p. 2021.11.15.468686).
https://doi.org/10.1101/2021.11.15.468686

Browning, B. L., \& Browning, S. R. (2009). A unified approach to
genotype imputation and haplotype-phase inference for large data sets of
trios and unrelated individuals. \emph{American Journal of Human
Genetics}, \emph{84}(2), 210--223.

Browning, B. L., \& Browning, S. R. (2013). Improving the accuracy and
efficiency of identity-by-descent detection in population data.
\emph{Genetics}, \emph{194}(2), 459--471.

Browning, S. R., \& Browning, B. L. (2007). Rapid and accurate haplotype
phasing and missing-data inference for whole-genome association studies
by use of localized haplotype clustering. \emph{American Journal of
Human Genetics}, \emph{81}(5), 1084--1097.

Browning, S. R., \& Browning, B. L. (2011). Haplotype phasing: existing
methods and new developments. \emph{Nature Reviews. Genetics},
\emph{12}(10), 703--714.

Burri, R. (2017). Interpreting differentiation landscapes in the light
of long‐term linked selection. \emph{Evolution Letters}.
https://onlinelibrary.wiley.com/doi/abs/10.1002/evl3.14

Cao, J., Schneeberger, K., Ossowski, S., Günther, T., Bender, S., Fitz,
J., Koenig, D., Lanz, C., Stegle, O., Lippert, C., Wang, X., Ott, F.,
Müller, J., Alonso-Blanco, C., Borgwardt, K., Schmid, K. J., \& Weigel,
D. (2011). Whole-genome sequencing of multiple Arabidopsis thaliana
populations. \emph{Nature Genetics}, \emph{43}(10), 956--963.

Carmi, S., Palamara, P. F., Vacic, V., Lencz, T., Darvasi, A., \& Pe'er,
I. (2013). The Variance of Identity-by-Descent Sharing in the
Wright--Fisher Model. \emph{Genetics}, \emph{193}(3), 911--928.

Castro, J. P. L., Yancoskie, M. N., Marchini, M., Belohlavy, S.,
Hiramatsu, L., Kučka, M., Beluch, W. H., Naumann, R., Skuplik, I., Cobb,
J., Barton, N. H., Rolian, C., \& Chan, Y. F. (2019). An integrative
genomic analysis of the Longshanks selection experiment for longer limbs
in mice. \emph{ELife}, \emph{8}, e42014.

Cheng, H., Concepcion, G. T., Feng, X., Zhang, H., \& Li, H. (2021).
Haplotype-resolved de novo assembly using phased assembly graphs with
hifiasm. \emph{Nature Methods}, \emph{18}(2), 170--175.

Clark, A. G. (2004). The role of haplotypes in candidate gene studies.
\emph{Genetic Epidemiology}, \emph{27}(4), 321--333.

Crawford, D. C., \& Nickerson, D. A. (2005). Definition and clinical
importance of haplotypes. \emph{Annual Review of Medicine}, 56,
303--320.

Davies, R. W., Flint, J., Myers, S., \& Mott, R. (2016). Rapid genotype
imputation from sequence without reference panels. \emph{Nature
Genetics}, \emph{48}(8), 965--969.

Davies, R. W., Kucka, M., Su, D., Shi, S., Flanagan, M., Cunniff, C. M.,
Chan, Y. F., \& Myers, S. (2021). Rapid genotype imputation from
sequence with reference panels. \emph{Nature Genetics}, \emph{53}(7),
1104--1111.

Delaneau, O., Zagury, J.-F., Robinson, M. R., Marchini, J. L., \&
Dermitzakis, E. T. (2019). Accurate, scalable and integrative haplotype
estimation. \emph{Nature Communications}, 10(1), 5436.

Eggertsson, H. P., Jonsson, H., Kristmundsdottir, S., Hjartarson, E.,
Kehr, B., Masson, G., Zink, F., Hjorleifsson, K. E., Jonasdottir, A.,
Jonasdottir, A., Jonsdottir, I., Gudbjartsson, D. F., Melsted, P.,
Stefansson, K., \& Halldorsson, B. V. (2017). Graphtyper enables
population-scale genotyping using pangenome graphs. \emph{Nature
Genetics}, 49(11), 1654--1660.

Fisher, R. A. (1954). A fuller theory of ``Junctions'' in inbreeding.
\emph{Heredity}, \emph{8}(2), 187--197.

Griffiths, R. C., \& Marjoram, P. (1997). An ancestral recombination
graph. \emph{Institute for Mathematics and Its Applications}, \emph{87},
257.

Grossman, S. R., Shylakhter, I., Karlsson, E. K., Byrne, E. H., Morales,
S., Frieden, G., Hostetter, E., Angelino, E., Garber, M., Zuk, O.,
Lander, E. S., Schaffner, S. F., \& Sabeti, P. C. (2010). A Composite of
Multiple Signals Distinguishes Causal Variants in Regions of Positive
Selection. \emph{Science}, \emph{327}(5967), 883--886.

Hartl, D. L., Clark, A. G., \& Clark, A. G. (1997). \emph{Principles of
population genetics} (Vol. 116). Sinauer associates Sunderland, MA.

Hellenthal, G., Busby, G. B. J., Band, G., Wilson, J. F., Capelli, C.,
Falush, D., \& Myers, S. (2014). A genetic atlas of human admixture
history. \emph{Science}, \emph{343}(6172), 747--751.

Hickey, G., Heller, D., Monlong, J., Sibbesen, J. A., Sirén, J.,
Eizenga, J., Dawson, E. T., Garrison, E., Novak, A. M., \& Paten, B.
(2020). Genotyping structural variants in pangenome graphs using the vg
toolkit. \emph{Genome Biology}, \emph{21}(1), 35.

Howie, B., Marchini, J., \& Stephens, M. (2011). Genotype imputation
with thousands of genomes. \emph{G3} , \emph{1}(6), 457--470.

Howie, B. N., Donnelly, P., \& Marchini, J. (2009). A flexible and
accurate genotype imputation method for the next generation of
genome-wide association studies. \emph{PLoS Genetics}, \emph{5}(6),
e1000529.

Hubisz, M. J., Williams, A. L., \& Siepel, A. (2020). Mapping gene flow
between ancient hominins through demography-aware inference of the
ancestral recombination graph. \emph{PLoS Genetics}, \emph{16}(8),
e1008895.

Hudson, R. R. (1983). Properties of a neutral allele model with
intragenic recombination. \emph{Theoretical Population Biology},
\emph{23}(2), 183--201.

Hudson, Richard R. (1990). Gene genealogies and the coalescent process.
\emph{Oxford Surveys in Evolutionary Biology}, \emph{7}(1), 44.

International HapMap Consortium. (2005). A haplotype map of the human
genome. \emph{Nature}, \emph{437}(7063), 1299--1320.

Jones, F. C., Grabherr, M. G., Chan, Y. F., Russell, P., Mauceli, E.,
Johnson, J., Swofford, R., Pirun, M., Zody, M. C., White, S., Birney,
E., Searle, S., Schmutz, J., Grimwood, J., Dickson, M. C., Myers, R. M.,
Miller, C. T., Summers, B. R., Knecht, A. K., \ldots{} Kingsley, D. M.
(2012). The genomic basis of adaptive evolution in threespine
sticklebacks. \emph{Nature}, \emph{484}(7392), 55--61.

Kelleher, J., Wong, Y., Wohns, A. W., Fadil, C., Albers, P. K., \&
McVean, G. (2019). Inferring whole-genome histories in large population
datasets. \emph{Nature Genetics}, \emph{51}(9), 1330--1338.

Kingman, J. F. C. (1982). The coalescent. \emph{Stochastic Processes and
Their Applications}, \emph{13}(3), 235--248.

Lawson, D. J., Hellenthal, G., Myers, S., \& Falush, D. (2012).
Inference of population structure using dense haplotype data. \emph{PLoS
Genetics}, \emph{8}(1), e1002453.

Leitwein, M., Duranton, M., Rougemont, Q., Gagnaire, P.-A., \&
Bernatchez, L. (2020). Using Haplotype Information for Conservation
Genomics. \emph{Trends in Ecology \& Evolution}, \emph{35}(3), 245--258.

Lewis, J.J., Geltman, R.C., Pollak, P.C., Rondem, K.E., Van Belleghem,
S.M., Hubisz, M.J., Munn, P.R., Zhang, L., Benson, C., Mazo-Vargas, A.
and Danko, C.G., 2019. Parallel evolution of ancient, pleiotropic
enhancers underlies butterfly wing pattern mimicry. \emph{Proceedings of
the National Academy of Sciences}, 116(48), pp.24174-24183.

Li, H., \& Ralph, P. (2019). Local PCA Shows How the Effect of
Population Structure Differs Along the Genome. \emph{Genetics},
\emph{211}(1), 289--304.

Li, N., \& Stephens, M. (2003). Modeling linkage disequilibrium and
identifying recombination hotspots using single-nucleotide polymorphism
data. \emph{Genetics}, \emph{165}(4), 2213--2233.

Li, Y., Willer, C. J., Ding, J., Scheet, P., \& Abecasis, G. R. (2010).
MaCH: using sequence and genotype data to estimate haplotypes and
unobserved genotypes. \emph{Genetic Epidemiology}, \emph{34}(8),
816--834.

Lohse, K., Chmelik, M., Martin, S. H., \& Barton, N. H. (2016).
Efficient Strategies for Calculating Blockwise Likelihoods Under the
Coalescent. \emph{Genetics}, \emph{202}(2), 775--786.

Lundberg, M., Liedvogel, M., Larson, K., Sigeman, H., Grahn, M., Wright,
A., Åkesson, S., \& Bensch, S. (2017). Genetic differences between
willow warbler migratory phenotypes are few and cluster in large
haplotype blocks. Evolution Letters, \emph{1}(3), 155--168.

Marchini, J., Howie, B., Myers, S., McVean, G., \& Donnelly, P. (2007).
A new multipoint method for genome-wide association studies by
imputation of genotypes. \emph{Nature Genetics}, \emph{39}(7), 906--913.

Martin, S. H., \& Van Belleghem, S. M. (2017). Exploring Evolutionary
Relationships Across the Genome Using Topology Weighting.
\emph{Genetics}, \emph{206}(1), 429--438.

Maynard Smith, J. M., \& Haigh, J. (1974). The hitch-hiking effect of a
favourable gene. \emph{Genetical Research}, \emph{23}(1), 23--35.

McVean, G. A. T., \& Cardin, N. J. (2005). Approximating the coalescent
with recombination. \emph{Philosophical Transactions of the Royal
Society of London. Series B, Biological Sciences}, \emph{360}(1459),
1387--1393.

Meier, J. I., Salazar, P. A., Kučka, M., Davies, R. W., Dréau, A.,
Aldás, I., Box Power, O., Nadeau, N. J., Bridle, J. R., Rolian, C.,
Barton, N. H., McMillan, W. O., Jiggins, C. D., \& Chan, Y. F. (2021).
Haplotype tagging reveals parallel formation of hybrid races in two
butterfly species. \emph{Proceedings of the National Academy of Sciences
of the United States of America}, \emph{118}(25).
https://doi.org/10.1073/pnas.2015005118

Mészáros, G., Milanesi, M., Ajmone-Marsan, P., \& Utsunomiya, Y. T.
(2021). Editorial: Haplotype analysis applied to livestock genomics.
\emph{Frontiers in Genetics}, \emph{12}, 660478.

Novembre, J., \& Barton, N. H. (2018). Tread Lightly Interpreting
Polygenic Tests of Selection. \emph{Genetics}, \emph{208}(4),
1351--1355.

Otte, K. A., \& Schlötterer, C. (2021). Detecting selected haplotype
blocks in evolve and resequence experiments. \emph{Molecular Ecology
Resources}, \emph{21}(1), 93--109.

Patil, N., Berno, A. J., Hinds, D. A., Barrett, W. A., Doshi, J. M.,
Hacker, C. R., Kautzer, C. R., Lee, D. H., Marjoribanks, C., McDonough,
D. P., Nguyen, B. T., Norris, M. C., Sheehan, J. B., Shen, N., Stern,
D., Stokowski, R. P., Thomas, D. J., Trulson, M. O., Vyas, K. R.,
\ldots{} Cox, D. R. (2001). Blocks of limited haplotype diversity
revealed by high-resolution scanning of human chromosome 21.
\emph{Science}, \emph{294}(5547), 1719--1723.

Poelstra, J. W., Vijay, N., Bossu, C. M., Lantz, H., Ryll, B., Müller,
I., Baglione, V., Unneberg, P., Wikelski, M., Grabherr, M. G., \& Wolf,
J. B. W. (2014). The genomic landscape underlying phenotypic integrity
in the face of gene flow in crows. \emph{Science}, \emph{344}(6190),
1410--1414.

Price, A. L., Tandon, A., Patterson, N., Barnes, K. C., Rafaels, N.,
Ruczinski, I., Beaty, T. H., Mathias, R., Reich, D., \& Myers, S.
(2009). Sensitive detection of chromosomal segments of distinct ancestry
in admixed populations. \emph{PLoS Genetics}, \emph{5}(6), e1000519.

Ralph, P., Thornton, K., \& Kelleher, J. (2020). Efficiently Summarizing
Relationships in Large Samples: A General Duality Between Statistics of
Genealogies and Genomes. \emph{Genetics}, \emph{215}(3), 779--797.

Rasmussen, M. D., Hubisz, M. J., Gronau, I., \& Siepel, A. (2014).
Genome-wide inference of ancestral recombination graphs. \emph{PLoS
Genetics}, \emph{10}(5), e1004342.

Ravinet, M., Faria, R., Butlin, R. K., Galindo, J., Bierne, N.,
Rafajlović, M., Noor, M. A. F., Mehlig, B., \& Westram, A. M. (2017).
Interpreting the genomic landscape of speciation: a road map for finding
barriers to gene flow. \emph{Journal of Evolutionary Biology},
\emph{30}(8), 1450--1477.

Richardson, J. L., Brady, S. P., Wang, I. J., \& Spear, S. F. (2016).
Navigating the pitfalls and promise of landscape genetics.
\emph{Molecular Ecology}, \emph{25}(4), 849--863.

Rockman, M. V. (2012). The QTN program and the alleles that matter for
evolution: all that's gold does not glitter. \emph{Evolution;
International Journal of Organic Evolution}, \emph{66}(1), 1--17.

Sabeti, P. C., Reich, D. E., Higgins, J. M., Levine, H. Z. P., Richter,
D. J., Schaffner, S. F., Gabriel, S. B., Platko, J. V., Patterson, N.
J., McDonald, G. J., Ackerman, H. C., Campbell, S. J., Altshuler, D.,
Cooper, R., Kwiatkowski, D., Ward, R., \& Lander, E. S. (2002).
Detecting recent positive selection in the human genome from haplotype
structure. \emph{Nature}, \emph{419}(6909), 832--837.

Sabeti, P. C., Varilly, P., Fry, B., Lohmueller, J., Hostetter, E.,
Cotsapas, C., Xie, X., Byrne, E. H., McCarroll, S. A., Gaudet, R.,
Schaffner, S. F., Lander, E. S., International HapMap Consortium,
Frazer, K. A., Ballinger, D. G., Cox, D. R., Hinds, D. A., Stuve, L. L.,
Gibbs, R. A., \ldots{} Stewart, J. (2007). Genome-wide detection and
characterization of positive selection in human populations.
\emph{Nature}, \emph{449}(7164), 913--918.

Schwartz, R., Halldórsson, B. V., Bafna, V., Clark, A. G., \& Istrail,
S. (2003). Robustness of inference of haplotype block structure.
\emph{Journal of Computational Biology: A Journal of Computational
Molecular Cell Biology}, \emph{10}(1), 13--19.

Sella, G., \& Barton, N. H. (2019). Thinking About the Evolution of
Complex Traits in the Era of Genome-Wide Association Studies.
\emph{Annual Review of Genomics and Human Genetics}, \emph{20},
461--493.

Sousa, V. C., Grelaud, A., \& Hey, J. (2011). On the nonidentifiability
of migration time estimates in isolation with migration models.
\emph{Molecular Ecology}, \emph{20}(19), 3956--3962.

Speidel, L., Forest, M., Shi, S., \& Myers, S. R. (2019). A method for
genome-wide genealogy estimation for thousands of samples. \emph{Nature
Genetics}, \emph{51}(9), 1321--1329.

Stankowski, S., Chase, M. A., Fuiten, A. M., Rodrigues, M. F., Ralph, P.
L., \& Streisfeld, M. A. (2019). Widespread selection and gene flow
shape the genomic landscape during a radiation of monkeyflowers.
\emph{PLoS Biology}, \emph{17}(7), e3000391.

Stankowski, S., \& Streisfeld, M. A. (2015). Introgressive hybridization
facilitates adaptive divergence in a recent radiation of monkeyflowers.
\emph{Proceedings. Biological Sciences / The Royal Society},
\emph{282}(1814). https://doi.org/10.1098/rspb.2015.1666

Steinrücken, M., Kamm, J., Spence, J. P., \& Song, Y. S. (2019).
Inference of complex population histories using whole-genome sequences
from multiple populations. \emph{Proceedings of the National Academy of
Sciences of the United States of America}, \emph{116}(34), 17115--17120.

Steinrücken, M., Spence, J. P., Kamm, J. A., Wieczorek, E., \& Song, Y.
S. (2018). Model-based detection and analysis of introgressed
Neanderthal ancestry in modern humans. \emph{Molecular Ecology},
\emph{27}(19), 3873--3888.

Stephens, M., \& Scheet, P. (2005). Accounting for decay of linkage
disequilibrium in haplotype inference and missing-data imputation.
\emph{American Journal of Human Genetics}, \emph{76}(3), 449--462.

Sundquist, A., Fratkin, E., Do, C. B., \& Batzoglou, S. (2008). Effect
of genetic divergence in identifying ancestral origin using HAPAA.
\emph{Genome Research}, \emph{18}(4), 676--682.

Szpiech, Z. A., \& Hernandez, R. D. (2014). selscan: an efficient
multithreaded program to perform EHH-based scans for positive selection.
\emph{Molecular Biology and Evolution}, \emph{31}(10), 2824--2827.

Taliun, D., Gamper, J., \& Pattaro, C. (2014). Efficient haplotype block
recognition of very long and dense genetic sequences. \emph{BMC
Bioinformatics}, \emph{15}, 10.

Tavares, H., Whibley, A., Field, D. L., Bradley, D., Couchman, M.,
Copsey, L., Elleouet, J., Burrus, M., Andalo, C., Li, M., Li, Q., Xue,
Y., Rebocho, A. B., Barton, N. H., \& Coen, E. (2018). Selection and
gene flow shape genomic islands that control floral guides.
\emph{Proceedings of the National Academy of Sciences of the United
States of America}, \emph{115}(43), 11006--11011.

Thompson, E. A. (2013). Identity by descent: variation in meiosis,
across genomes, and in populations. \emph{Genetics}, \emph{194}(2),
301--326.

Todesco, M., Owens, G. L., Bercovich, N., Légaré, J.-S., Soudi, S.,
Burge, D. O., Huang, K., Ostevik, K. L., Drummond, E. B. M., Imerovski,
I., Lande, K., Pascual-Robles, M. A., Nanavati, M., Jahani, M., Cheung,
W., Staton, S. E., Muños, S., Nielsen, R., Donovan, L. A., \ldots{}
Rieseberg, L. H. (2020). Massive haplotypes underlie ecotypic
differentiation in sunflowers. \emph{Nature}, \emph{584}(7822),
602--607.

Turner, I., Garimella, K. V., Iqbal, Z., \& McVean, G. (2018).
Integrating long-range connectivity information into de Bruijn graphs.
\emph{Bioinformatics} , \emph{34}(15), 2556--2565.

Voight, B. F., Kudaravalli, S., Wen, X., \& Pritchard, J. K. (2006). A
map of recent positive selection in the human genome. \emph{PLoS
Biology}, \emph{4}(3), e72.

Wallbank, W. R., Baxter S.W., Pardo-Diaz, C., Hanly, J. J., Martin, S.
H., Mallet, J., Dasmahapatra, K.K., et al. (2016) "Evolutionary novelty
in a butterfly wing pattern through enhancer shuffling." \emph{PLoS
biology} 14: e1002353.

Wakeley, J. (2009). \emph{Coalescent theory :an introduction /}
(575:519.2 WAK). sidalc.net.
http://www.sidalc.net/cgi-bin/wxis.exe/?IsisScript=FCL.xis\&method=post\&formato=2\&cantidad=1\&expresion=mfn=010195

Wakeley, J., \& Wilton, P. R. (2016). \emph{Coalescent and models of
identity by descent}.

Wallberg, A., Schöning, C., Webster, M. T., \& Hasselmann, M. (2017).
Two extended haplotype blocks are associated with adaptation to high
altitude habitats in East African honey bees. \emph{PLoS Genetics},
\emph{13}(5), e1006792.

Weisenfeld, N. I., Kumar, V., Shah, P., Church, D. M., \& Jaffe, D. B.
(2017). Direct determination of diploid genome sequences. \emph{Genome
Research}, \emph{27}(5), 757--767.

Whitlock, M. C., \& Mccauley, D. E. (1999). Indirect measures of gene
flow and migration: FST≠1/(4Nm+1). \emph{Heredity}, \emph{82}(2),
117--125.

Wohns, A. W., Wong, Y., Jeffery, B., Akbari, A., Mallick, S., Pinhasi,
R., Patterson, N., Reich, D., Kelleher, J., \& McVean, G. (2021). A
unified genealogy of modern and ancient genomes. In \emph{bioRxiv} (p.
2021.02.16.431497). https://doi.org/10.1101/2021.02.16.431497

Wolf, J. B. W., \& Ellegren, H. (2017). Making sense of genomic islands
of differentiation in light of speciation. \emph{Nature Reviews.
Genetics}, \emph{18}(2), 87--100.

Zhang, K., Calabrese, P., Nordborg, M., \& Sun, F. (2002). Haplotype
block structure and its applications to association studies: power and
study designs. \emph{American Journal of Human Genetics}, \emph{71}(6),
1386--1394.
\end{quote}

\end{document}
